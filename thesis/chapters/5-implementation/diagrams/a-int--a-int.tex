\begin{figure}
\centering

\begin{tikzpicture}[scale=2,
type/.style = {
    rectangle, rounded corners,
    draw=black, fill=blue!40, drop shadow,
    text centered, anchor=north, text=white, text width=2.6cm,
},
arrow/.style = {
    shorten >= 0.33cm, shorten <= 0.33cm
}]

\node (root)  [type, text width=6cm] at (0, 2) {FunctionType\\(a $\rightarrow$ Integer) $\rightarrow$ a $\rightarrow$ Integer};
\node (lFunc) [type] at (-1.8, 1) {FunctionType\\a $\rightarrow$ Integer};
\node (rFunc) [type] at (1.8, 1) {FunctionType\\a $\rightarrow$ Integer};
\node (lA)    [type] at (-2.7, 0) {UnboundType\\a};
\node (rA)    [type] at (0.9, 0) {UnboundType\\a};
\node (lInt)  [type] at (-0.9, 0) {LiteralType\\Integer};
\node (rInt)  [type] at (2.7, 0) {LiteralType\\Integer};

\draw[->, arrow] (root) -- (lFunc);
\draw[->, arrow] (root) -- (rFunc);
\draw[->, arrow] (lFunc) -- (lA);
\draw[->, arrow] (lFunc) -- (lInt);
\draw[->, arrow] (rFunc) -- (rA);
\draw[->, arrow] (rFunc) -- (rInt);

\end{tikzpicture}

\caption{The internal representation of: (a $\rightarrow$ Integer) $\rightarrow$ a $\rightarrow$ Integer}
\label{fig:a-int--a-int}
    
\end{figure}