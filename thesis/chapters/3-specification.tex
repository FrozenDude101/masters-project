\section{Specification} \label{specification}
A variety of measures will be used to determine if the project is a success. To ensure a robust system, these will be implemented incrementally, and tested for reliability and correctness, as well as usability by the end user.

\subsection{Functional Requirements} \label{functional-requirements}
\begin{enumerate}
    \item The tool must be able to parse user input into the correct expression, and parse as they would expect a Haskell expression to.
    \item The tool should report appropriate errors on malformed inputs, including both syntax and type errors, so the user can debug and modify their code.
    \item The tool must be able to evaluate the expression correctly, and evaluate as they would expect a Haskell expression to.
    \item The tool must be able to visualise expressions correctly, to accurately represent the expression.
    \item The tool must be able to visualise the type of expressions correctly, to accurately represent the type of the expression.
\end{enumerate}

\subsection{Non-Functional Requirements} \label{non-functional-requirements}
\begin{enumerate}
    \item The tool should parse user inputs within a short period of time, so the user can quickly see changes they make take effect.
    \item The tool should not slow down when visualising large expressions, as Haskell expressions can become very large while they evaluate.
    \item The tool should be easy to understand by the user, to ensure they can learn from it.
\end{enumerate}