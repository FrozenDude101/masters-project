\section{Survey Results}
\label{survey}

\begin{table}[h]
    \centering
    \begin{tabular}{p{0.6\linewidth}|p{0.25\linewidth}}
        Question & Responses\\\hline
        Are/have you taken the 2$^{nd}$ year Functional Programming module? & 11 - Yes, 3 - Currently taking it.\\\hline
        For Haskell, have you used any tools/visualisers for testing and debugging? & 10 - No, 3 - GHCi, 1 - VSCode\\\hline
        Which tools/visualisers have you used for testing and debugging other languages? & 5 - VSCode Plugins, 4 - IDE Debuggers, 2 - GDB, 3 - No\\\hline
        How did you find the transition from object-oriented imperative languages to a functional language? (1-5, hard-easy) & 1 2 2 3 3 3 3\newline4 4 4 4 4 5 5\\\hline
        How well would you say you understood the following concepts? (0-1-5, I don't know what that is-not well-very well) &\\\hline
        Higher Order Functions &2 3 4 4 4 4 4\newline4 5 5 5 5 5 5\\\hline
        Anonymous Functions &3 3 3 4 4 4 4\newline5 5 5 5 5 5 5\\\hline
        Monads &0 0 1 2 2 2 3\newline3 3 3 3 4 4 4\\\hline
        Lazy Evaluation &0 0 0 0 3 3 3\newline4 4 4 4 5 5 5\\\hline
        Partial Application &0 1 2 2 3 3 4\newline4 4 4 5 5 5 5\\\hline
        Pure Functions &0 0 1 1 2 3 3\newline4 4 4 5 5 5 5\\
    \end{tabular}
    \caption{Questions and responses as a preliminary survey.}
    \label{tab:user-evaluation}
\end{table}

\begin{table}[h]
    \centering
    \begin{tabular}{p{\linewidth}}\hline
    It was sometimes unclear what exactly designated say a certain class of functions (eg for me, monads) - I could use them, but never fully understand WHY they worked or were a certain way. I tried a lot to mostly just practise lots of different applications of different concepts to get an intuitive feel for them.\\\hline
    Monads were slightly tricky. Visualising them as containers/wrappers around some primitive type, and the bind operator as a transform from one wrapper to another, helped a lot\\\hline
    I mainly found a disconnect between how the concepts were explained and how I would actually apply them within programs\\\hline
    Mainly the lack of examples that we could apply the concepts to. With lazy evaluation and monads, I found that it wasn't until I went away and did my own learning in my free time (with textbooks and videos) that they made total sense and I could use them.\\\hline
    I understand monads at a surface level but not much more. This may be because we weren’t required to learn them that deeply but I still didn’t find them intuitive\\\hline
    Why are functions pure and impure? I never really figured it out. I remember memorising some functions that "lifted" functions to return a value, but that has left my brain now.\\\hline
    \end{tabular}
    \caption{Additional comments to the ``How well would you say you understood the following concepts?'' question.}
    \label{tab:comments}
\end{table}

\begin{table}[h]
    \centering
    \begin{tabular}{p{\linewidth}}\hline
        Yes, mostly because I found them fun\\\hline
        Very likely because it is expressive for some tasks, esp those requiring correctness and notational accuracy with the underlying math\\\hline
        Not very likely because it has bad tooling (stack <<<<<)\\\hline
        I would be fairly likely as I find recursion intuitive and enjoy the elegance of functional programs\\\hline
        Extremely likely!\\\hline
        I just enjoy using functional languages so I'll use them for fun for personal projects, but also I want to go into research and lecturing in a Theory of Computation department, so they can prove very useful\\\hline
        Fairly likely, since I've worked with them in industry\\\hline
        I’m much more comfortable in oop languages so not likely, but if a use case comes up that is perfect for functional I would.\\\hline
        Unlikely. Most jobs don’t require skills related to functional programming\\\hline
        Pretty likely, it seems like an interesting concept that I enjoy programming in the style of.\\\hline
        Python is my main programming language and inherits many aspects of functional programming that I use frequently: list comprehensions, currying \& partial functions, Anonymous/lambda functions, higher order functions.\\\hline
        Less likely as it's not that interesting to me\\\hline
        Unlikely - however functional paradigms are something I use in imperative software as a result of having studied purely functional languages.\\\hline
        Honestly, never again\\\hline
    \end{tabular}
    \caption{Answers to the ``How likely are you to use a functional language in the future?'' question.}
    \label{tab:future}
\end{table}